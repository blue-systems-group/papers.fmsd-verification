\section{Prerequisites and Problem Statement}
\label{sec:background}

\subsection{Packet, Trace and Monitor State Machine}
\label{subsec:basic}

The alphabet of the monitor state machine is the finite set of all valid packets
defined by the protocol, denoted as $\mathbb{P}$.  A packet is a binary string
of a finite number of bits, encoding interesting protocol attributes such as
\texttt{src}, \texttt{dest}, \texttt{type}, \texttt{flags}, and physical layer
information, such as \texttt{channel}, \texttt{modulation}, etc. The input of
the state machine then corresponds to a time-ordered sequence of packets.

\begin{definition}[Packet Trace]
  \sloppy{
  A \textit{packet trace} is a finite sequence of $(timestamp, packet)$ tuple:
  $[(t_1, p_1), (t_2, p_2),\ldots,(t_n, p_n)]$ where $t_i \in \mathbb{Z}^+$ is
  the \textit{discrete} timestamp and $p_i$ is the packet observed at time
  $t_i$. The timestamps are strictly monotonically increasing: $t_i < t_{i+1}$
  for $1 \le i < n$.
}
\end{definition}%

In addition to timestamp monotonicity, we also require that adjacent packets do
not overlap in time, $t_{i+1}-t_i > \text{\texttt{airtime}}(p_i)$ for $1 \le i <
n$, where \texttt{airtime()} calculates the time taken to transmit a packet.
The timestamp represents the observer's local clock ticks, and need not to be
synchronized among devices.

We use \textit{timed automata}~\cite{alur1994theory} to model the expected
behaviors of the DUT. A timed automata is a finite state machine with timing
constraints on the transitions: each transition can optionally start one or more
timers, which can later be used to assert certain events should be seen before
or after the time out event. We refer the readers to~\cite{alur1994theory} for
more details about timed automata.


\begin{definition}[Monitor]
  \sloppy{%
    A protocol monitor state machine $S$ is a 7-tuple
    $\{\Sigma, \mathbb{S}, \mathbb{X}, s_0, C, E, G\}$, where:
  }
  \begin{itemize}
    \item $\Sigma = \mathbb{P}$ is the finite input alphabet.

    \item $\mathbb{S}$ is a non-empty, finite set of states. $s_0 \in
      \mathbb{S}$ is the initial state.

    \item $\mathbb{X}$ is the set of boolean variables. We use $v = \{x
      \leftarrow true/false\ |\ x \in \mathbb{X}\}$ to denote an assignment of
      the variables. Let $\mathbb{V}$ be the set of such values $v$. 

    \item $C$ is the set of clock variables. A \textit{clock variable} can be
      reset along any state transitions. At any instant, reading a clock
      variable returns the time elapsed since last time it was reset.

    \item $G$ is the set of guard conditions defined inductively by
      \begin{align*}
        g := true\ |\ c \le T\ |\ c \ge T\ |\ x\ |\ \neg g\ |\ g_1 \land g_2
      \end{align*}%
      where $c \in C$ is a clock variable, $T$ is a constant, and $x$ is a
      variable in $\mathbb{X}$.  A transition can choose not to use guard
      conditions by setting $g$ to be $true$.

    \item $E \subseteq \mathbb{S} \times \mathbb{V} \times \mathbb{S} \times
      \mathbb{V} \times \Sigma \times  G \times \mathscr{P}(C)$ gives the set of
      transitions.\\ $\langle s_i, v_i, s_j, v_j, p, g, C'\rangle \in E$
      represents that if the monitor is in state $s_i$ with variable assignments
      $v_i$, given the input tuple $(t, p)$ such that the guard $g$ is
      satisfied, the monitor can transition to a state $s_j$ with variable assignments
      $v_j$, and reset the clocks in $C'$ to 0.
  \end{itemize}
  \label{def:sm}
\end{definition}

A tuple $(t_i, p_i)$ in the packet trace means the packet $p_i$ is presented to
the state machine at time $t_i$.  As an example, the monitor state machine
illustrated in Fig.~\ref{fig:dot11_tx_ta} can be formally defined as follows:

\begin{itemize}
  \item $\Sigma = \{\it{DATA}_i, \it{DATA}'_i, Ack\ |\ 0 \le i < N\}$.
  \item Clock variables $C = \{c\}$. The only clock variable $c$ is
    used for acknowledgment time out.
  \item $\mathbb{X} = \{i\}$, as a variable with ${\mathit log}(N) + 1$ bits to
    count from $0$ to $N$.  
  \item Guard constraints $G = \{ c \le T_o, c > T_o, T_o < c \le T_m\}$.
    $T_o$ is the acknowledgment time out value, and $T_m >
    T_o$ is the maximum delay allowed before the retransmission packet gets
    sent. In this particular case, $T_o$ can be arbitrary large but not infinity
        in order to check the liveness of the DUT.
\end{itemize}


\begin{definition}[Trace Rejection and Acceptance]
    A monitor state machine $S$ {\it rejects a trace} $Tr$ if there
    exists a prefix of $Tr$ such that all states reachable after consuming the
    prefix have no valid transitions for the next $(t, p)$ input. Otherwise, $S$
    {\it accepts} $Tr$ (or $Tr$ {\it satisfies} $S$).
\end{definition}


The monitor state machine defines a \textit{timed language} $L$ which consists
of all valid packet traces that can be observed by the DUT.  We now give the
definition of protocol \textit{compliance} and \textit{violation}.

\begin{definition}[Violation and Compliance]
  Suppose $\mathbb{T}$ is the set of all possible packet traces collected from
  DUT, and $S$ is the state machine specified by the protocol. The DUT
  \textit{violates} the protocol specification if there exists an
  packet trace $Tr \in \mathbb{T}$ such that $S$ rejects $Tr$.
  Otherwise, the DUT is \textit{compliant} with the specification.
\end{definition}

The focus of this paper is to determine whether a \textit{given} sniffer trace
$Tr$ is evidence of a violation.
%
%We acknowledge that determining \textit{compliance} is a more challenging
%problem, as it requires enumerating every possible trace in $\mathbb{T}$, which
%is probably infinite.


\subsection{Mutation Trace}
\label{subsec:mutation}

As shown in the motivation example in Fig.~\ref{fig:sniffer_in_middle}, a
sniffer trace may either miss packets that are present in DUT trace, or contain
extra packets that are missing in DUT trace. Note that in the latter case, those
extra packets must be all sent \textit{to} the DUT. This is because it is
impossible for the sniffer to overhear packets sent from the DUT that were not
actually sent by the DUT.

We formally capture this relationship with the definition of mutation trace.

\begin{definition}[Mutation Trace]
  \label{def:mutation}
  A packet trace $Tr'$ is a mutation of sniffer trace $Tr$ w.r.t a DUT if for
  all $(t, p) \in Tr \setminus Tr'$, $p.dest = DUT$, where $p.dest$ is the
  destination of packet $p$.
\end{definition}

By definition, either $Tr' \supseteq Tr$ (hence $Tr \setminus Tr' = \emptyset$),
or those extra packets in $Tr$ but not in $Tr'$ are all sent to the DUT. Note
that $Tr'$ may contain extra packets that are either sent to or received by
the DUT.

A mutation trace $Tr'$ represents a \textit{guess} of the corresponding DUT
packet trace given sniffer trace $Tr$.  In fact, the DUT packet trace must
be one of the mutation traces of the sniffer trace $Tr$.

\begin{lemma}
  Let $Tr_{DUT}$ and $Tr$ be the DUT and sniffer packet trace captured during
  the same protocol operation session, and $\mathcal{M}(Tr)$ be the set of
  mutation traces of $Tr$ with respect to DUT, then $Tr_{DUT} \in \mathcal{M}(Tr)$.
  \label{lem:mutation}
\end{lemma}%
\begin{proof}
  Let $\Delta = Tr \setminus Tr_{DUT}$ be the set of packets that are in $Tr$
  but not in $Tr_{DUT}$. Recall that it is not possible for the sniffer to
  observe packets sent from the DUT that the DUT did not send. Therefore,
  all packets in $\Delta$ are sent \textit{to} the DUT. That is, for
  all $(t, p) \in \Delta,\ p.dest = DUT$. By Definition~\ref{def:mutation},
  $Tr_{DUT} \in \mathcal{M}(Tr)$.
  \qed
\end{proof}


\subsection{Problem Statement}
\label{subsec:problem}

Lemma~\ref{lem:mutation} shows that $\mathcal{M}(Tr)$ is a \textit{complete} set
of guesses of the DUT packet trace. Therefore, the problem of validating DUT
implementation given a sniffer trace can be formally defined as follows:

\begin{problem}
  \label{prob:validation}
  VALIDATION\\
  \textbf{instance} A protocol monitor state machine $S$ and a sniffer trace $Tr$.\\
  \textbf{question} Does there exist a mutation trace $Tr'$ of $Tr$ that satisfies $S$?
\end{problem}

If the answer is no, a definite violation of the DUT implementation can be
claimed. Nevertheless, if the answer is yes, $S$ may still reject $Tr_{DUT}$.
In other words, the conclusion of the validation can either be
\textit{definitely wrong} or \textit{probably correct}, but not
\textit{definitely correct}.  This is the fundamental limitation caused by the
uncertainty of sniffer traces.
