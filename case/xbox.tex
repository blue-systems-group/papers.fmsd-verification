\subsection{Industrial Application}
\label{subsec:xbox}

\begin{table}[t!]
  \centering
  \begin{tabular}{lll}
    \toprule
    \textbf{Protocol Aspects} & \textbf{Traces} & \textbf{Violations (\%)}\\
    \midrule
    Sequence Number & 3049 & 1539 (50.48\%) \\
    Station Scheduling & 3046 & 2045 (67.14\%) \\
    Uplink Modulation & 3127 & 8 (0.26\%) \\
    Downlink Modulation & 3127 & 24 (0.77\%) \\
    \bottomrule
  \end{tabular}
  \caption{\textbf{Validation Results on Traces from the Gaming Controller Wireless Protocol.}}
  \label{tab:xbox}
\end{table}

We have applied our framework for runtime verification of the wireless protocol
implementation of a commercial product that has several millions of shipping
devices. The product runs a proprietary wireless protocol to reduce latency and
power consumption. Sniffer traces were collected in regular testing process, but
were only manually inspected previously. 

We obtained 75 sniffer traces from the testing team for a new version of the
protocol that is under development and testing. This team has been testing the
implementation for a few weeks.  Each trace contained around 6 million packets
that were captured during 1 hour and 40 minutes of protocol operation.

We first split the traces into \num{100000} packet segments, which yields \num{3127}
traces for testing.  We then applied our framework on the traces to validate the
DUT implementation. We found that the latest implementation of the protocol
under development had violations of the protocol specification.  Some of the
implementation bugs we found related to the sequence number management, station
scheduling during power saving mode, and modulation rates adaptation.
Table~\ref{tab:xbox} summarizes the validation results. The \textbf{Traces}
column shows the number of traces that we have successfully validated for each
category, and the third column shows the fraction of traces that contain at
least one violation in that category.

Note that if we disable
the augmented transitions, each trace will be flagged as violation because of
the missing packets, thereby reducing the usability of the tool. We also note
that some bugs manifest more often than others. For instance, the bugs related
to packet sequence number and station scheduler were detected in about half the
traces, and the bug related to the rate control algorithm was detected in only a
few traces. This is because the previous two aspects are essential in all
protocol operations, while the bugs related to rate control only manifest
themselves under certain link conditions.

After communicating with the testing team, we confirmed that the sequence number
bug was already known, as it is relatively easy to detect even by manually
examining the traces. The bug related to station scheduling was also noticed
before, yet no quantitative results about how frequent this bug appears were
obtained because of the lack of automatized validating tools. Finally, the bug
related to rate control, which was unknown previously, has been filed as a bug
report. All reported bugs were fixed before the next release the product.
